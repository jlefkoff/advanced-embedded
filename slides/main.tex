\documentclass{beamer}

% -----------------------------------------------------------------------------
% Packages
% -----------------------------------------------------------------------------

\usepackage{graphicx}   % For including images (e.g., QR codes)
\usepackage{hyperref}   % For clickable hyperlinks
\usepackage{xcolor}     % For defining custom colors
\usepackage{iexec}     % Uncomment if you want to use Git commit hash for document version
\usepackage{tcolorbox}  % For creating colored boxes for organization and emphasis
\usepackage{qrcode}     % For generating QR codes directly in LaTeX
\usepackage{pifont}     % For using dingbat symbols (e.g., checkmarks for compatibility)

\newcommand{\gitAbbrevHash}{\iexec{git rev-parse --short HEAD}} % Command to get Git commit hash for versioning (requires iexec)

% -----------------------------------------------------------------------------
% Hyperlink Colors Configuration
% -----------------------------------------------------------------------------
% Sets colors for hyperlinks to improve readability
\hypersetup{
    colorlinks=true,   % Enables colored links instead of default boxed links
    linkcolor=blue,    % Color for internal document links (e.g., TOC)
    urlcolor=cyan      % Color for external links (e.g., URLs)
}

% -----------------------------------------------------------------------------
% Footer Configuration
% -----------------------------------------------------------------------------
% Customizes the footer to display organization, workshop name, semester, and page number
\setbeamertemplate{footline}{%
  \leavevmode%
  \hbox{
    % Left: Organization Name
    \begin{beamercolorbox}[wd=.35\paperwidth,ht=2.25ex,dp=1ex,leftskip=.3cm]{author in head/foot}%
      Wireless Club at Northeastern University%
    \end{beamercolorbox}%
    % Center: Workshop Name
    \begin{beamercolorbox}[wd=.20\paperwidth,ht=2.25ex,dp=1ex,center]{author in head/foot}%
      Adv. Embedded Workshop%
    \end{beamercolorbox}%
    % Center: Semester Information
    \begin{beamercolorbox}[wd=.15\paperwidth,ht=2.25ex,dp=1ex,center]{author in head/foot}%
      Fall 2024%
    \end{beamercolorbox}%
    % Right: Version (Git commit hash) and Page Number
    \begin{beamercolorbox}[wd=.30\paperwidth,ht=2.25ex,dp=1ex,rightskip=.3cm plus1fil]{author in head/foot}%
      Version: \gitAbbrevHash{} \hspace{1em} \insertframenumber/\inserttotalframenumber%
    \end{beamercolorbox}}%
  \vskip0pt%
}

% -----------------------------------------------------------------------------
% Slide: Title Page
% -----------------------------------------------------------------------------
\title{Advanced Embedded Development Workshop}
% \subtitle{Hosted by the Northeastern University Institute of Electrical and Electronics Engineers (IEEE) Club}
\author{Northeastern University Wireless Club - W1KBN}
\date{November 4, 2024}

\begin{document}

% -----------------------------------------------------------------------------
% Slide: Title with QR Code for Sign-In
% -----------------------------------------------------------------------------
\begin{frame}
    \titlepage % Generates the title, subtitle, author, and date
    \vspace{-1cm} % Adjusts spacing to accommodate the QR code
    \begin{center}
        \textbf{Sign in Here:} \\ % Prompt for participants to sign in
        % \includegraphics[width=0.30\textwidth]{images/nuwc-signin-qrcode.png} % Image-based QR code (if needed)
        \qrcode{https://l.w1kbn.org/signin} % Generates QR code for sign-in URL
        \\
        {\small \url{https://l.w1kbn.org/signin}} % Sign-in URL for participants to manually access
    \end{center}
\end{frame}

% -----------------------------------------------------------------------------
% Slide: General Information and Sign-In
% -----------------------------------------------------------------------------
\begin{frame}
    \begin{tcolorbox}[colframe=blue!75!black, colback=blue!10, title=About, center title]
        \begin{itemize}
            \item Regular Meetings: Thursdays, 7 PM @ Hayden Hall, Room 503
            \item Workshops: Mondays, 7 PM @ East Village, Room 010
        \end{itemize}        
    \end{tcolorbox}
    
    \begin{tcolorbox}[colframe=blue!75!black, colback=blue!10, title=QR Codes \& Links, center title]
        \begin{minipage}{0.32\textwidth}
            \centering
            \textbf{Sign in} \\
            \qrcode{https://l.w1kbn.org/signin} \\
            \tiny \url{https://l.w1kbn.org/signin}  % Smaller font for URL
        \end{minipage}
        \begin{minipage}{0.32\textwidth}
            \centering
            \textbf{Slack} \\
            \qrcode{https://neuwireless.slack.com/join/signup} \\
            \tiny \url{https://neuwireless.slack.com/join/signup}
        \end{minipage}
        \begin{minipage}{0.32\textwidth}
            \centering
            \textbf{Mailing List} \\
            \qrcode{http://eepurl.com/gduCIr} \\
            \tiny \url{http://eepurl.com/gduCIr}
        \end{minipage}
        \begin{itemize}
            \item Website: \url{https://nuwireless.org/}
        \end{itemize}
    \end{tcolorbox}
\end{frame}

% -----------------------------------------------------------------------------
% Slide: Workshop Goals
% -----------------------------------------------------------------------------
\begin{frame}
    \frametitle{Workshop Goals}
    \begin{itemize}
        \item Explore the fundamentals of bare-metal programming on the STM32 platform
        \item Gain experience with assembly language for low-level control
        \item Learn to interact with sensors and input devices at a hardware level
        \item Understand the role of the Hardware Abstraction Layer (HAL) for embedded C development
        \item Build a foundation for industry-relevant embedded development techniques
    \end{itemize}
\end{frame}

% -----------------------------------------------------------------------------
% Slide: Workshop Overview
% -----------------------------------------------------------------------------
\begin{frame}
    \frametitle{Workshop Overview}
    \begin{columns}
        \begin{column}{0.55\textwidth}
            \begin{itemize}
                \item Introduction to STM32 and bare-metal programming
                \item Hands-on practice with assembly to control hardware
                \item Setting up and using the embedded C HAL for streamlined development
                \item Working with sensors and input devices directly through registers
                \item Transitioning from low-level assembly to higher-level embedded C workflows
            \end{itemize}
        \end{column}
        \begin{column}{0.4\textwidth}
            \begin{center}
                \includegraphics[width=\textwidth]{images/nuwc-advanced-embedded-workshop-overview.jpg} % Placeholder image for visual interest
            \end{center}
        \end{column}
    \end{columns}
\end{frame}

% -----------------------------------------------------------------------------
% Slide: Bill of Materials (BOM)
% -----------------------------------------------------------------------------
\begin{frame}
    \frametitle{Bill of Materials (BOM)}
    \textbf{Kit Contents:}
    \begin{itemize}
        \item \textbf{Solderless Breadboard} (White, self-adhesive)
            \begin{itemize}
                \item Column separation: 0.3 inches
                \item Pin spacing: 0.1 inches
                \item Compatible wire sizes: 29-20 AWG
            \end{itemize}
        \item \textbf{LSM6DSOX dev board} – 6-axis IMU
        \item \textbf{NUCLEO-F031K6} – STM32 Nucleo-32 Development Board
        \item \textbf{USB C to USB Micro A}
        \item \textbf{QWIIC to Male Headers Cable}
        \item \textbf{LEDs} (4 total, any consistent color)
        \item \textbf{220$\Omega$} (4 total, through-hole)
        \item \textbf{Jumper Wires} (5 total)
        \item \textbf{Rotary Encoder}
    \end{itemize}
\end{frame}

% -----------------------------------------------------------------------------
% Slide: Software Requirements 1
% -----------------------------------------------------------------------------
\begin{frame}
    \frametitle{Software Requirements}
    \footnotesize
    \vspace{-0.5cm}
    \begin{table}[]
        \begin{tabular}{|p{2cm}|p{3cm}|p{1cm}|p{3.5cm}|}
            \hline
            \textbf{Component} & \textbf{Description} & \textbf{Version} & \textbf{Download Link} \\ \hline
            \textbf{VS Code} & IDE / Text Editor & Latest & \url{https://code.visualstudio.com/} \\ \hline
            \textbf{VS Code Extension: \texttt{Cortex-Debug}} & ARM Cortex-M GDB Debugger support for VS Code & 1.12.1 & \url{https://marketplace.visualstudio.com/items?itemName=marus25.cortex-debug} \\ \hline
            \textbf{GNU Arm Embedded Toolchain} & Compiler/toolchain: \texttt{gcc-arm-none-eabi} & 10.3-2021.10 & \url{https://developer.arm.com/downloads/-/gnu-rm} \\ \hline
            \textbf{CMake} & Build system & Latest & \url{https://cmake.org/download/} \\ \hline
			\textbf{GNU Make (Windows)} & Build system, should be already installed on Linux/MacOS & Latest & \url{https://gnuwin32.sourceforge.net/packages/make.htm} \\ \hline
        \end{tabular}
    \end{table}
\end{frame}

% -----------------------------------------------------------------------------
% Slide: Software Requirements 2
% -----------------------------------------------------------------------------
\begin{frame}
    \frametitle{Software Requirements}
    \footnotesize
    \vspace{-0.5cm}
    \begin{table}[]
        \begin{tabular}{|p{2cm}|p{3cm}|p{1cm}|p{3.5cm}|}
            \hline
            \textbf{Component} & \textbf{Description} & \textbf{Version} & \textbf{Download Link} \\ \hline
            \textbf{J-Link (Win / MacOS)} & Flashing and debugging software & V8.10e & \url{https://www.segger.com/downloads/jlink/} \\ \hline
            \textbf{ST-Link (Linux)} & Open source toolset for STM32 devices & Latest & Arch/Fedora: \texttt{stlink} or Debian: \texttt{stlink-tools} \\ \hline
            \textbf{Source Code} & Workshop code repo & Latest & \url{https://github.com/jlefkoff/advanced-embedded} \\ \hline
        \end{tabular}
    \end{table}
\end{frame}

% -----------------------------------------------------------------------------
% Slide: Theory (1) % TODO
% -----------------------------------------------------------------------------
\begin{frame}
    \frametitle{Theory (1)}
    \textbf{Opcode / What is Assembly}
    \begin{itemize}
        \item Placeholder for explanation of opcodes and assembly language basics
    \end{itemize}

    \textbf{Register/Stack}
    \begin{itemize}
        \item Placeholder for details about registers and their roles in assembly
    \end{itemize}

    \textbf{Program Counter (PC)}
    \begin{itemize}
        \item Placeholder for description of the Program Counter and its function
    \end{itemize}
\end{frame}

% -----------------------------------------------------------------------------
% Slide: Theory (2) % TODO
% -----------------------------------------------------------------------------
\begin{frame}
    \frametitle{Theory (2)}
    \textbf{Infinite Loop}
    \begin{itemize}
        \item Placeholder for explanation of infinite loops / their purpose
    \end{itemize}

    \textbf{Instruction Set Register (ISR) in Detail}
    \begin{itemize}
        \item Placeholder for detailed explanation of ISRs
    \end{itemize}
\end{frame}

% -----------------------------------------------------------------------------
% Slide: Theory (3) % TODO
% -----------------------------------------------------------------------------
\begin{frame}
    \frametitle{Theory (3)}
    \textbf{Subroutines}
    \begin{itemize}
        \item Placeholder for explanation of subroutines, how they work, and their importance in assembly
    \end{itemize}
\end{frame}

% -----------------------------------------------------------------------------
% Slide: Industry Use HAL %
% -----------------------------------------------------------------------------
\begin{frame}
    \frametitle{Industry Use - HAL}
    \begin{itemize}
      \item The HAL (hardware abstraction layer) is a layer of software that abstracts the hardware of a microcontroller.
      \item your time as a dev is more valuable than writing individual register/timer/DMA/SPI/UART code.
      \item HAL provides a higher level of abstraction that allows you to write code that is more portable across different microcontrollers.
    \end{itemize}
\end{frame}


% -----------------------------------------------------------------------------
% Slide: Peripheral Overview %
% -----------------------------------------------------------------------------
\begin{frame}
  \frametitle{IMU}
    \begin{itemize}
      \item The LSM6DSOX is a 6-axis IMU (Inertial Measurement Unit) that measures acceleration and angular velocity.
      \item We interface with the IMU using the I2C protocol (talked about it last week).
      \item It has a \dots library! Written by ST. Easy peasy!
    \end{itemize}
\end{frame}


% -----------------------------------------------------------------------------
% Slide: Contact Information
% -----------------------------------------------------------------------------
\begin{frame}
    \frametitle{Contact Us}
    \begin{itemize}
        \item Questions? Feel free to reach out!
        \item Workshop Team Emails: \\
        \{\href{mailto:elarbi.m@northeastern.edu}{elarbi.m}, 
        \href{mailto:aviedov.v@northeastern.edu}{aviedov.v}, 
        \href{mailto:heaney.ma@northeastern.edu}{heaney.ma}\}[at]northeastern[d0t]edu
        \item General Workshop Email: \href{mailto:workshops@nuwireless.org}{workshops}[at]nuwireless[d0t]org
        \item Website: \url{https://nuwireless.org/}
        \item Location: Hayden Hall, Room 503
    \end{itemize}
    \vspace{1cm}
    \begin{flushright}
        \footnotesize{© 2024 Northeastern Wireless Club} \\
        \footnotesize{Design: \href{https://melarbi.com}{Muhammad Elarbi}, based on LaTeX Beamer}
    \end{flushright}
\end{frame}

% -----------------------------------------------------------------------------
% End of Document
% -----------------------------------------------------------------------------
\end{document}
